\documentclass[]{article}

\usepackage{amsmath}
\usepackage[]{graphicx}
\usepackage{subfigure}
\usepackage[latin1]{inputenc}
\usepackage{comment}
\usepackage{url}
%\usepackage{biblatex} 
%\usepackage[pdftex]{graphicx}
\usepackage{anysize}
\marginsize{3cm}{3cm}{2cm}{2.5cm}%l r t b

\title{Face and Facial Expression Recognition}
%\title{Controlling a smart phone using gaze gestures}
%\title{Gaze gestures as an innovative HCI method for smartphone like devices}
 
\author{Eduardo Neiva, Jesus Nuevo, David Rozado}

\setlength\parindent{0pt} %no indentation in paragraphs
% Document starts
\begin{document}

\maketitle

\section{Abstract}
We present our work on an open source system to classify facial expressions from continuous video input.

We study person-dependent and person-independent recognition of identity and expression.

\section{Introduction}
Human computer interaction could be agumented by providing a computer systems with the ability to recognize the emotions
of the humans interfacing the computer. Emotions are conveyed by humans  using the visual, vocal and other physiological
means such as body gestures. Facial expressions are an indirect proxy to measure the internal cognitive emotions of
humans. Impaired facial expression recognition by humans can be a sign of serious cognitive dysfunction such as
schizophrenia \cite{Edwards2002789}.  Facial attributes can be tracked computationally through a video stream of the
subject's face. Making the computer aware of the emotions of the user could lead the way towards more natural interaction.


In this work we describe the usage of an open-source face tracker to feed a set of machine learning classifiers that the
tech subjects identity and their facial expressions. The face tracker provides the extraction of invariant features in
the tracked faces. We used support vector machines  to classify the invariant features provided by the face tracker.


The psychological tradition has usefully classify facial expressions in seven categories: neutral, anger, disgust, fear,
joy, sadness and surprise.


Previous work has employed different techniques to classify  facial expressions. The work from \cite{Cohen2003160}
tested the friend nation network classifiers for classifying expression from media, focusing on changes in distribution
assumptions and feature dependency structures. Authors also proposed an enough that the architecture of hidden Markov
models for automatically segmenting and recognizing human facial expressions. Authors reported recognition rates up to
83\% For frame-based recognition methods and 82\% for the multilevel HMM. 
 

The work from \cite{Chen670976} investigated the emotional contents of speech and video based facial expression to
proposes an bioinspired algorithm for human facial expression recognition, concluding that both modalities can be
complimentary and able to achieve higher recognition rates than either morality alone.


Authors in \cite{Bartlett4624313} use perceptual primitives to code 7 facial expressions in real-time. the system first
detects frontal faces using a cascade of feature detectors trained with boosting techniques. The expression recognizer
receives image patches located by the phase detector. A Gabor representation of the parts is used  by a bank of kernel
based classifiers. Authors used a combination of Adaboost and support vector machines to enhance performance. One of the
most interesting properties of these work was its ability to change the outputs of the classifiers smoothly as a
function of time, hence, providing a dynamical representation of facial expression.

The work from \cite{lijunyin} deviates from the typical 2D static image or 2D the video sequence recognition of
facial expression arguing that that a 2D-based analyses is incapable of handling large pose variations and proposing
instead the usage of classification techniques on 3-D facial expression models and making available to the community a
database of prototypical 3D facial expression shapes.
 
The work from \cite{Cohn840611} made available over 2000 digitized image sequences from  182 adult subjects of
varying ethnicity, performing multiple tokens of most primary facial expressions to create a comprehensive testbed for
comparative studies of facial expression analysis.

Authors in \cite{Busso:2004} also used a multimodal approach to combine acoustic information and facial expression
analysis in order to detect human emotions. In their work, authors demonstrated that when both modalities are fused, the performance
and the robustness of the emotion recognition system improves considerably.


Researchers have employed a variety of methods to carry out facial expression recognition such as optical flow
computation  and symbolic representations \cite{Yacoob506414}, local binary patterns \cite{Shan2009803},  Bayesian
network classifiers \cite{Cohen1211408}, geometric deformation features and support vector machines
\cite{kotsia4032815}, hidden Markov models \cite{aleksic1597130, Cohen2003160}, Parametric flow models
\cite{blackAndYacoob}, AdaBoost and linear discriminant analysis \cite{bartlett1398364}. The surveys from
\cite{bartlett1398364} and \cite{Fasel2003259} are two good review sources about machine learning methods 
for fully automatic recognition of facial expressions. 


In summary, in this paper we apply \ldots for identity recognition of facial expression recognition bla bla bla.



\section{Methodology}
\subsection{Face Tracker}

\subsection{Experimental Setup}
This is a reference to Figure \ref{figureLabel}

\begin{figure}[ht]
\begin{center}
\vspace{-3mm}
\includegraphics[width=0.95\textwidth,height=50mm]{figures/fileName.jpg}
\end{center}
\caption{\textbf{Title of caption.} Explanation of the figure.}
\label{figureLabel}
\end{figure}


\section{Results}
More plain text.

\section{Summary and Discussion}
In this work we have used an open-source face tracker to classify facial expressions.


\bibliographystyle{plain}
\bibliography{library}

\end{document}
